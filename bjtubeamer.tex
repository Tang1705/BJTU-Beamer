\documentclass[a4paper,zihao=-4,notitlepage,oneside,openright]{ctexart}

\usepackage{geometry}
\geometry{left=2.5cm,right=2.5cm,top=3.0cm,bottom=3.0cm,headheight=1.5cm,footskip=1.75cm}

\usepackage[toc]{multitoc}
\usepackage{caption}
\usepackage{floatrow}
\usepackage{float}
\usepackage{threeparttable}

\usepackage{xcolor}
\definecolor{bjtudarkblue}{HTML}{10307E}

\usepackage{fontspec}
\setmainfont{Times New Roman}

\usepackage{booktabs}
\renewcommand{\arraystretch}{1.2}


\let\oldverbatim=\verbatim
\let\endoldverbatim=\endverbatim
\renewenvironment{verbatim}{\begin{oldverbatim}\vspace{-0.5cm}\color{red}\textbf}{\vspace{-0.3cm}\end{oldverbatim}\ignorespacesafterend}


\title{\textbf{\heiti\color{bjtudarkblue}B{\zihao{-4}JTU}B{\zihao{-4}EAMER}:北京交通大学(非官方)\\Beamer 主题}}
\author{Q.~Tang\\{\zihao{-4}qitang@bjtu.edu.cn}}
\date{\zihao{-4}v1.0.0 (2022/05/20)}

\ctexset{
	abstractname = {\heiti\zihao{-4} 摘要},
	contentsname = {\centering\zihao{-4} \textbf{目录}\\},
    section = {
        format=\raggedright\heiti\zihao{-3},
        beforeskip={24pt},
        afterskip={18pt},
    },
    subsection = {
        format=\raggedright\heiti\zihao{4},
        beforeskip={24pt},
        afterskip={18pt},
    },
    paragraph = {
    	format = \centering\heiti\zihao{-4}\textbf\\,
    	beforeskip={32pt},
    	afterskip={24pt},
    },
}

\usepackage{titletoc}
\titlecontents{section}[3em]{\hspace{0em}}{\songti\contentslabel{2em}}{\songti \zihao{-4}}{\titlerule*[0.3pc]{.}\contentspage}[\vspace{0ex}]

\DeclareCaptionFont{bjtusongti}{\zihao{5}\songti}
\captionsetup{font=bjtusongti}

\DeclareFloatFont{bjtusongti}{\zihao{5}\songti}
\DeclareFloatVCode{tableafterfloat}{\vspace{-4pt}}
\floatsetup[table]{font=bjtusongti,capposition=top,captionskip=2pt,postcode=tableafterfloat}

\begin{document}
\maketitle

\abstract{B{\zihao{-6}JTU}B{\zihao{-6}EAMER} 主题为作者基于本科毕业设计中使用的 PPT 模板制作,旨在提供对应风格的 Beamer 制作方式,可用于学位答辩、课堂演示、学术交流或其他需要演示文稿的活动,方便地使用 \LaTeX{} 制作含有学校特色的演示文稿。}

\paragraph{使用说明}
\begin{enumerate}
	\item 本主题为作者个人制作,非官方模板,含有学校特色的主题素材版权归北京交通大学所有, 使用仅供参考, 任何由于使用本主题而引起的任何问题均与本主题作者无关。
	\item 任何个人或组织以本模板为基础进行修改、扩展而生成的新的专用模板, 请严格遵守有关协议。由于违犯协议而引起的任何纠纷争端均与本模板作者无关。
	\item 由于作者水平有限,当前版本重在实现原 PPT 模板的风格,欢迎喜欢该主题风格的朋友基于当前版本改进。
\end{enumerate}

\tableofcontents

\setcounter{page}{0}
\thispagestyle{empty}
\newpage

\section{模板介绍}


BJTUBEAMER (Beijing Jiaotong University \LaTeX{} Beamer Template) 是为了尽可能地还原作者在本科毕业设计中使用的 PPT 模板风格,从而可以方便地使用 \LaTeX{} 进行制作,本文档将对该模板的使用方法进行简单介绍,非常欢迎对 Beamer 主题制作有更深层次了解的朋友对本主题的设计、开发做出贡献。
下表列出了 BJTUBEAMER 的主要文件及其介绍:


\begin{table}[!htbp]
\centering
\caption{B{\zihao{-6}JTU}B{\zihao{-6}EAMER} 文件及其介绍表}
\begin{threeparttable}
\begin{tabular}{c|c}
\toprule[2pt]
文件(夹)&描述	\\\hline
beamerthemebjtubeamer.sty&演示主题,控制演示稿的的每一细节的外观\\\hline
beamerinnerthemebjtubeamer.sty&内部主题,控制着排版演示稿的什么元素\\\hline
beamercolorthemebjtubeamer.sty&色彩主题,设置幻灯片各部分的配色\\\hline
beamerfontthemebjtubeamer.sty&字体主题,控制着演示稿使用的字体的属性\\\hline
resources/&主题相关素材存储文件夹\\\hline
figures/&用户制作演示文稿图片存储文件夹\\\hline
main.tex&主题测试文件\\\hline
bjtubeamer.pdf&用户手册(本文档)\\\bottomrule[2pt]
\end{tabular}
\begin{tablenotes}
     \item[1] 使用前,请仔细阅读 bjtubeamer.pdf,即本文档。
   \end{tablenotes}
   \end{threeparttable}
\end{table}

\section{使用说明}

\begin{verbatim}
	\title{<arg1>}
\end{verbatim}
中文标题

\begin{verbatim}
	\englishtitle{<arg1>}
\end{verbatim}
英文标题

\begin{verbatim}
	\author{<arg1>}
\end{verbatim}
作者/答辩人

\begin{verbatim}
	\studentNumber{<arg1>}
\end{verbatim}
学号

\begin{verbatim}
	\advisor{<arg1>}
\end{verbatim}
指导教师

\begin{verbatim}
	\institution{<arg1>}
\end{verbatim}
所在学院/实验室

\begin{verbatim}
	\defense{<arg1>}
\end{verbatim}
Beamer 所用场合,学位答辩/课堂展示

\begin{verbatim}
	\maketitle
\end{verbatim}
生成封面页

\begin{verbatim}
	\makecontent
\end{verbatim}
生成目录页

\begin{verbatim}
	\makebackcover
\end{verbatim}
生成封底页

\begin{verbatim}
	\section{<arg1>}
\end{verbatim}
对 LATEX 的命令进行了重新定义,除了保留原有的功能外,同时会生成一个新的章节封面,改封面同时会展示当前章节的 subsection 结构。

\begin{verbatim}
	\background{<arg1>}
\end{verbatim}
设置当前 frame 的背景图片

\end{document}